%
% 6.006 problem set 1 solutions template
%
\documentclass[12pt,twoside]{article}

\usepackage{amsmath}
\usepackage{color}

\input{macros}

\setlength{\oddsidemargin}{0pt}
\setlength{\evensidemargin}{0pt}
\setlength{\textwidth}{6.5in}
\setlength{\topmargin}{0in}
\setlength{\textheight}{8.5in}

\newcommand{\theproblemsetnum}{3}
\newcommand{\releasedate}{March 16, 2017}
\newcommand{\duedate}{April 6, 2017}
\newcommand{\tabUnit}{3ex}
\newcommand{\tabT}{\hspace*{\tabUnit}}

\title{6.006 Problem Set 3}

\begin{document}

\handout{Problem Set \theproblemsetnum}{March 16, 2017}

\textbf{All parts are due {\bf \duedate} at {\bf 11:59PM}}.

\setlength{\parindent}{0pt}

\medskip

\hrulefill

\medskip

{\bf Name:} Caroline Pech

\medskip

{\bf Collaborators:} Carly Staub

\medskip

\hrulefill

%%%%%%%%%%%%%%%%%%%%%%%%%%%%%%%%%%%%%%%%%%%%%%%%%%%%%
% See below for common and useful latex constructs. %
%%%%%%%%%%%%%%%%%%%%%%%%%%%%%%%%%%%%%%%%%%%%%%%%%%%%%

% Some useful commands:
%$f(x) = \Theta(x)$
%$T(x, y) \leq \log(x) + 2^y + \binom{2n}{n}$
% {\tt code\_function}


% You can create unnumbered lists as follows:
%\begin{itemize}
%    \item First item in a list 
%        \begin{itemize}
%            \item First item in a list 
%                \begin{itemize}
%                    \item First item in a list 
%                    \item Second item in a list 
%                \end{itemize}
%            \item Second item in a list 
%        \end{itemize}
%    \item Second item in a list 
%\end{itemize}

% You can create numbered lists as follows:
%\begin{enumerate}
%    \item First item in a list 
%    \item Second item in a list 
%    \item Third item in a list
%\end{enumerate}

% You can write aligned equations as follows:
%\begin{align} 
%    \begin{split}
%        (x+y)^3 &= (x+y)^2(x+y) \\
%                &= (x^2+2xy+y^2)(x+y) \\
%                &= (x^3+2x^2y+xy^2) + (x^2y+2xy^2+y^3) \\
%                &= x^3+3x^2y+3xy^2+y^3
%    \end{split}                                 
%\end{align}

% You can create grids/matrices as follows:
%\begin{align}
%    A = 
%    \begin{bmatrix}
%        A_{11} & A_{21} \\
%        A_{21} & A_{22}
%    \end{bmatrix}
%\end{align}

\begin{problems}

\section*{Part A}

\problem  % Problem 1
Submit this to gradescope.

\begin{problemparts}
\problempart Part a  % Problem 1a
\problempart Part b  % Problem 1b
\end{problemparts}

\problem  Submit this to gradescope.% Problem 2
\begin{problemparts}
\problempart Part a  % Problem 2a
\problempart Part b  % Problem 2b
\end{problemparts}

\problem  Submit this to gradescope. % Problem 3

\begin{problemparts}
\problempart Part a % Problem 3a
\problempart Part b % Problem 3b
\end{problemparts}

\problem Submit this to gradescope. % Problem 4

\begin{problemparts}
\problempart Part a % Problem 4a
\problempart Part b % Problem 4b
\problempart Part c % Problem 4c
\problempart Part d % Problem 4d
\problempart Part e % Problem 4e
\problempart Part f % Problem 4f
\problempart Part g % Problem 4g
\problempart Part h % Problem 4h
\problempart Part i % Problem 4i
\problempart Part j % Problem 4j
\end{problemparts}

\section*{Part B}

\problem
Submit your implementation on \url{alg.csail.mit.edu}.

\end{problems}

\end{document}

